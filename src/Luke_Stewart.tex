
%%%%%%%%%%%%%%%%%%%%%%%%%%%% Document Setup %%%%%%%%%%%%%%%%%%%%%%%%%%%%
\documentclass[8pt]{article}

% This is a helpful package that puts math inside length specifications
\usepackage{calc}
% \usepackage{biblatex}
\usepackage[bibstyle=publist,style=authortitle-comp,backend=biber,doi=false,giveninits=true,maxcitenames=99,isbn=false,url=false]{biblatex}
\addbibresource{10.1007_978-3-031-40132-9_17-citation.bib}
\addbibresource{m20.bib}
\addbibresource{citation-12413_195.bib}
\addbibresource{citation-13355_137.bib}
\addbibresource{citation-13355_170.bib}
\AtEveryBibitem{\clearfield{month}}
\AtEveryCitekey{\clearfield{month}}
\AtEveryBibitem{\clearname{editor}}
\AtEveryCitekey{\clearname{editor}}
\AtEveryBibitem{\clearlist{location}}
\AtEveryCitekey{\clearlist{location}}
% \plauthorname{Stewart}
% \usepackage[
%   backend      = biber,
%   doi          = false,
%   giveninits   = false,
%   maxcitenames = 99,
%   isbn         = false,
%   url          = false,
% ]{biblatex}
% \usepackage{fontspec}
% \usepackage{fontawesome}

% Layout: Puts the section titles on left side of page
\reversemarginpar

%% Use these lines for letter-sized paper
\usepackage[paper=letterpaper,
            %includefoot, % Uncomment to put page number above margin
            marginparwidth=1.2in,     % Length of section titles
            marginparsep=.05in,       % Space between titles and text
            margin=1in,               % 1 inch margins
            includemp]{geometry}

%% Use these lines for A4-sized paper
%\usepackage[paper=a4paper,
%            %includefoot, % Uncomment to put page number above margin
%            marginparwidth=30.5mm,    % Length of section titles
%            marginparsep=1.5mm,       % Space between titles and text
%            margin=25mm,              % 25mm margins
%            includemp]{geometry}

%% More layout: Get rid of indenting throughout entire document
\setlength{\parindent}{0in}

%% This gives us fun enumeration environments. compactenum will be nice.
\usepackage{paralist}

%% Reference the last page in the page number
%
% NOTE: comment the +LP line and uncomment the -LP line to have page
%       numbers without the ``of ##'' last page reference)
%
% NOTE: uncomment the \pagestyle{empty} line to get rid of all page
%       numbers (make sure includefoot is commented out above)
%
\usepackage{fancyhdr,lastpage}
\pagestyle{fancy}
%\pagestyle{empty}      % Uncomment this to get rid of page numbers
\fancyhf{}\renewcommand{\headrulewidth}{0pt}
\fancyfootoffset{\marginparsep+\marginparwidth}
\newlength{\footpageshift}
\setlength{\footpageshift}
          {0.5\textwidth+0.5\marginparsep+0.5\marginparwidth-2in}
\lfoot{\hspace{\footpageshift}%
       \parbox{4in}{\, \hfill %
                    \arabic{page} of \protect\pageref*{LastPage} % +LP
%                    \arabic{page}                               % -LP
                    \hfill \,}}

% Finally, give us PDF bookmarks
\usepackage{color,hyperref}
\definecolor{darkblue}{rgb}{0.0,0.0,0.3}
\hypersetup{colorlinks,breaklinks,
            linkcolor=darkblue,urlcolor=darkblue,
            anchorcolor=darkblue,citecolor=darkblue}

%%%%%%%%%%%%%%%%%%%%%%%% End Document Setup %%%%%%%%%%%%%%%%%%%%%%%%%%%%


%%%%%%%%%%%%%%%%%%%%%%%%%%% Helper Commands %%%%%%%%%%%%%%%%%%%%%%%%%%%%

% The title (name) with a horizontal rule under it
%
% Usage: \makeheading{name}
%
% Place at top of document. It should be the first thing.
\newcommand{\makeheading}[1]%
        {\hspace*{-\marginparsep minus \marginparwidth}%
         \begin{minipage}[t]{\textwidth+\marginparwidth+\marginparsep}%
                {\large \bfseries #1}\\[-0.15\baselineskip]%
                 \rule{\columnwidth}{1pt}%
         \end{minipage}}

% The section headings
%
% Usage: \section{section name}
%
% Follow this section IMMEDIATELY with the first line of the section
% text. Do not put whitespace in between. That is, do this:
%
%       \section{My Information}
%       Here is my information.
%
% and NOT this:
%
%       \section{My Information}
%
%       Here is my information.
%
% Otherwise the top of the section header will not line up with the top
% of the section. Of course, using a single comment character (%) on
% empty lines allows for the function of the first example with the
% readability of the second example.
\renewcommand{\section}[2]%
        {\pagebreak[2]\vspace{1.\baselineskip}%
         \phantomsection\addcontentsline{toc}{section}{#1}%
         \hspace{0in}%
         \marginpar{
         \raggedright \scshape #1}#2}

% An itemize-style list with lots of space between items
% \newenvironment{outerlist}[1][\enskip\textbullet]%
%         {\begin{enumerate}[#1]}{\end{enumerate}%
%          \vspace{-.3\baselineskip}}
\newenvironment{outerlist}[1][\enskip\textbullet]%
       {\begin{compactenum}[#1]}{\end{compactenum}%
        \vspace{-.3\baselineskip}}
% An itemize-style list with little space between items
\newenvironment{innerlist}[1][\enskip\textbullet]%
        {\begin{compactenum}[#1]}{\end{compactenum}}

% To add some paragraph space between lines.
% This also tells LaTeX to preferably break a page on one of these gaps
% if there is a needed pagebreak nearby.
\newcommand{\blankline}{\quad\pagebreak[2]}
\def\CC{{C\nolinebreak[4]\hspace{-.05em}\raisebox{.4ex}{\tiny\bf ++}}}
%%%%%%%%%%%%%%%%%%%%%%%% End Helper Commands %%%%%%%%%%%%%%%%%%%%%%%%%%%

%%%%%%%%%%%%%%%%%%%%%%%%% Begin CV Document %%%%%%%%%%%%%%%%%%%%%%%%%%%%

\begin{document}
        \makeheading{Luke C.~Stewart}

        \section{\textbf{Contact Information}}
%
% NOTE: Mind where the & separators and \\ breaks are in the following
%       table.
%
% ALSO: \rcollength is the width of the right column of the table
%       (adjust it to your liking; default is 1.85in).
%
                \newlength{\rcollength}\setlength{\rcollength}{1.85in}%
                \begin{tabular}[t]{@{}p{\textwidth-\rcollength}p{\rcollength}}
                        \href{https://www.google.com/maps/place/8079+Quince+Cir,+Centennial,+CO+80112}{8079 Quince Cir.}          & \textit{Cell\,\,\,\,\,\,\,:} (256) 683-1560 \\
                        \href{https://www.google.com/maps/place/8079+Quince+Cir,+Centennial,+CO+80112}{Centennial, CO 80112}      & \textit{E-mail:} \href{mailto:stewalc@gmail.com}{stewalc@gmail.com}\\
                \end{tabular}


        \section{\textbf{Security Clearance}}
                Department of Defense \textbf{Secret}
                \blankline
                \blankline
                \blankline

        \section{\textbf{Education}}
                \href{http://www.auburn.edu/}{\textbf{Auburn University}} -- Auburn, Alabama USA\\
                \blankline
                B.S., Electrical Engineering, May 2006.
                % \begin{outerlist}
                        % \item B.S., Electrical Engineering, May 2006.
                        
                % \item
                % G.P.A. $3.8\,/\,4.0$
                %         \href{http://www.eng.auburn.edu/}
                %              {Electrical Engineering}, May 2006\\

                %         \begin{innerlist}
                %         \item Electrical specialization (emphasis on electromagnetics and control systems)
                %         \end{innerlist}

                % \end{outerlist}

        % \section{\textbf{Objective}}
        %         {\small Software/Systems Engineering position with hardware and software integration/simulation
        %         activities in a dynamic company offering  multiple technical and leadership growth opportunities
        %         }

        \section{\textbf{Summary of Qualifications}}
                \begin{outerlist}
                {
                        \small
                        \item[] \textbf{Intangibles}
                        \begin{innerlist}
                                \item An energetic and self-motivated engineer with a record of successfully leading groups,
                                organizing tasks and designing, implementing, deploying, and maintaining full-stack solutions ahead of schedule and under budget
                                \item Effective communicator across cultural, organizational, and engineering disciplines
                                % \item Experienced at prioritizing and multitasking to complete large projects
                                % \item Team player with strong work ethic; organized \& goal oriented
                        \end{innerlist}

                        \blankline
                        \item[] \textbf{Technical Skills}
                        %
                        \begin{innerlist}
                                \item Programming: 
                                        Python,
                                        C/\CC,
                                        Java,
                                        Bash, 
                                        Go,
                                        JavaScript (React, Angular, Svelte),
                                        Awk,
                                        Groovy,
                                        \LaTeX{},
                                        Markdown,
                                        PyTorch, NLTK, MCP, LLM, Ollama

                                % \item Programming: 
                                %         Awk,
                                        % Basic,
                                        % Bash, 
                                        % C/\CC,
                                        % Go,
                                        % Groovy,
                                        % Html,
                                        % Java,
                                        % JavaScript,
                                        % \LaTeX{},
                                        % Markdown,
                                        % Perl,
                                        % Python
                                        % Sed,
                                        % Tcl/Tk,
                                        % Xml

                                % \blankline 
                                \item Applications: 
                                        AMPCS,
                                        Ant, 
                                        AWS, 
                                        % Chef, 
                                        % ClearCase, 
                                        % ClearQuest, 
                                        Claude,
                                        CMake, 
                                        % CVS, 
                                        Docker/Podman, 
                                        % DOORS, 
                                        Eclipse, 
                                        % Emacs, 
                                        FPrime,
                                        Git (Hub, Lab, command line), 
                                        Gradle, 
                                        GreenHills,
                                        Ivy, 
                                        Jenkins,
                                        Linux/Unix platforms, 
                                        % Mathcad, 
                                        % Mathematica, 
                                        Matlab, 
                                        NGINX,
                                        % Microsoft Office, 
                                        % NASA WorldWind, 
                                        Octave, 
                                        Purify,
                                        % Rational Rhapsody, 
                                        % Rational Rose, 
                                        % Rational System Architect, 
                                        SQL (many DB flavors),
                                        Sun Grid Engine (SGE), 
                                        Subversion, 
                                        System ToolKit (STK), 
                                        Terraform, 
                                        Valgrind, 
                                        Vim, 
                                        VMWare, 
                                        VSCode, 
                                        VxWorks
                        \end{innerlist}
                }
                \end{outerlist}

                % \blankline

        \section{\textbf{Professional Experience}}
                %
                \href{https://www.jpl.nasa.gov/}{\textbf{NASA Jet Propulsion Laboratory}} -- Pasadena, California USA
                        \begin{outerlist}
                        \item[] \textit{Software Systems Engineer III}%
                                \hfill \textbf{May 2019 to present}
                        \begin{innerlist}
                                \item Currently working in Mission Control Information Systems (393C).
                                \item Cognizant Engineer (CogE) -- \href{https://dsocmct.jpl.nasa.gov/}{Psyche Deep Space Optical Communications (DSOC) OpenMCT}. Locally deployed (Podman containers) at Ground Laser Transmit (GLT) and Ground Laser Receive (GLR) Antennas with real-time synchronized telemetry to AWS deployed (GovCloud; JPL Net) custom OpenMCT adaptation with RethinkDB (real-time) and InfluxDB (time series; historical) data management running with automated telemetry processing and replication. The AWS RESTFul endpoint provides necessary real-time Psyche pass telemetry observation for all mission stakeholders.
                                \item \href{https://github.jpl.nasa.gov/MUPA/fmux}{Multiple Uplink Per Antenna (MUPA) Space Link Extension (SLE) Forward CLTU (FCLTU) Service Multiplexer (FMUX)}. Python-based novel approach for multiplexing CLTUs allowing one DSN antenna to service multiple spacecrafts' uplink needs.  
                                \item Common Mission Control (CMC). Integrating \href{https://github.com/yamcs/yamcs}{YAMCS} mission control framework (Java) into JPL missions to establish a NASA-wide Common Mission Control Ground Data System (GDS), which requires a deep understanding of CCSDS packets, frames, and Space Link Extension (SLE) interfaces
                                \item \href{https://github.jpl.nasa.gov/stewartl/Pass_Automation}{Psyche Pass Automation}. Python program currently used by Psyche, Europa Clipper, and other Earth-orbiting missions to automatically initiate space antenna station connection and downlink processing using AMPCS driven by DSN (SPS) track scheduling. Supports passes on the DSN in addition to the NEN and SN.

                                \item Actively supported Mars 2020 (Perseverance Rover; Ingenuity Drone) automated execution and analysis of Second Chance (SECC) FSW to ensure
                                successful Martian Entry Descent and Landing (EDL). This uses my framework for automated and distributed execution of FSW simulation
                                across 50 servers to complete 450 individual test cases.

                                \item Created a self-contained automated Docker orchestration framework to launch and interact with AMPCS
                                (Ground Control Software), Vista (a NASA OpenMCT plugin for telemetry visualization), and cFS (Core Flight
                                System; an open-source flight software framework)/FPrime (a flight-proven, multi-platform, open-source flight software framework) to demonstrate Mission Control System's full suite of
                                functionality with a simple, user-facing emphasis.

                                \item Helped implement a Continuous Integration pipeline for Mars 2020 CS3 (Common Software and Services
                                Subsystem) to build Docker images and auto-deploy AWS (Amazon Web Services) GovCloud environments for
                                80+ subsystem component functionalities using Jenkins, Docker, and terraform.\\

                        \end{innerlist}
                \end{outerlist}

                \href{http://www.raytheon.com/}{\textbf{Raytheon Company}} -- Pasadena, California USA
                \begin{outerlist}
                        \item[] \textit{Principal Software Engineer}%
                                \hfill \textbf{June 2017 to April 2019}
                        \begin{innerlist}
                                \item Developed, Executed, and Reviewed Functional Integration Tests (FIT) for Mars 2020 Flight Software.
                                % \item Implemented updates to existing Python Test Framework (PTF) to run the simulation in “split” mode (distributed across 2 servers; RHEL5 and RHEL7).
                                % \item (Also fixed the need for a“split” solution while implementing it – provided an all-RHEL7 runtime solution)
                                % \item Concurrently adapted the “split” and “RHEL7-unified” solutions for NISAR, SMAP, Psyche, and Europa Clipper Missions (for some Multi-Mission design/architecture experience)
                                \item Architected and implemented a Continuous Integration Automated Testing Solution for Mars 2020 (WART).\\
                        \end{innerlist}
                \end{outerlist}

                \href{http://www.raytheon.com/}{\textbf{Raytheon Company}} -- Aurora, Colorado USA
                \begin{outerlist}
                        \item[] \textit{Systems Engineer}%
                                \hfill \textbf{August 2010 to June 2017}
                        \begin{innerlist}
                                \item GPS OCX automation/integration specialist, Subject matter expert for JPL program code (RTGx Kalman Filter; Trajedy astrodynamics propagator).
                                \item GPS OCX System Simulator (GSYS) developer, integrator, and subject matter expert.
                                \item Worked with Digital Defense Services (DDS) to integrate DevOps mentality and practices into the GPS OCX Program.
                                % \item Successfully extended python with the JPL \CC\ libraries (developed necessary \CC\ API wrappers).
                                \item Implemented Distributed Computing framework for analysis/execution using parallel processing (\CC\ [an IPC methodology]) and MapReduce (modular Distributed Computing) techniques. 
                                % Software Engineer -- August 2010 to September 2015
                                % \item Developer for GSYS. Streamlined Build/Package/Deploy Process for GSYS.
                                \item Wrapped real-time Satellite Flight Software (\CC\ and ADA compiled with GreenHills and GNAT Pro/AdaCore, respectively) to interact within the GSYS Simulation Framework.
                                % \item Integrator for GSYS and GPS OCX Operational Baseline (MCS). Played critical role in driving the OCX operational software with GSYS.
                                % \item Developed and implemented the Space Vehicle Dynamics Model (SVDM). An API to the JPL GPS OCX astrodynamics algorithms (\CC) used by all simulated Space Vehicle Types in GSYS.
                                % \item Developed and implemented the United States Naval Observatory (USNO) External Interface model to generate and consume NGA GPS products.
                                % \item Developed many utilities on personal time to solve Program Risks (e.g., Automated SW Inventory, DiffSLOC, MP)\\
                        \end{innerlist}
                \end{outerlist}

                \href{http://www.raytheon.com/}{\textbf{Raytheon Company}} -- Woburn, Massachusetts USA
                \begin{outerlist}
                        \item[] \textit{Systems Engineer}%
                                \hfill \textbf{July 2008 to August 2010}
                        \begin{innerlist}
                                \item Support Analyst at MDIOC in Colorado Springs as AN/TPY-2 radar simulation (CRUSHM) support analyst for the TA-10 event.
                                \item Worked on classified defense contract tasks for simulation, modeling, and analysis of Forward Based X-Band - Transportable (FBX-T) and AN/TPY-2 Radar systems.
                                % \item Worked on development and maintenance of CRUSHM radar simulation (\CC).
                                \item Prepared, compiled, and installed various releases of CRUSHM on-site at customer locations.
                                % \item Designed, ahead of schedule and underbudget, a Software Design Document (SDD) by creating an automated documentation generation tool.
                                % \item Helped administer and operate a distributed Linux computing cluster built to expedite CRUSHM radar simulation, Monte Carlo analyses, and genetic algorithm studies.
                                % \item Gained valuable insight into the procedural approach to designing/engineering a large scale \CC simulation product -- using UML methodologies -- on a timeline for a government customer.
                                % \item Represented my company successfully in engineering design, integration, and support activities conducted in the government customer's classified labs on Redstone Arsenal, AL, and MDIOC Schriever AFB, Colorado Springs, CO.
                                % \item Interacted professionally with government customers while hosting simulation training classes.\\
                        \end{innerlist}
                \end{outerlist}

                % \newpage
                % \href{http://www.dese.com/}{\textbf{DESE Research, Inc.}} -- Huntsville, Alabama USA
                % \begin{outerlist}
                % \item[] \textit{Electrical Engineer}%
                %         \hfill \textbf{June 2006 to July 2008}
                % \begin{innerlist}
                % \item Worked on classified defense contract tasks for simulation, modeling, and analysis of missile systems in a 6 Degrees of Freedom (6DOF) environment as well as TOW Missile Hardware In the Loop (HWIL) simulations lab to develop wireless ``sensor to shooter" linkages.
                % \item Gained extremely useful software knowledge and ability to include Python scripting/automation and \CC model development.
                % \item Designed, ahead of schedule and under-budget, a Control Actuation System open-loop simulation and analysis toolkit.
                % \item Helped design, construct, implement, and operate a distributed computing cluster built solely from excessed PCs and open source software.
                % \end{innerlist}
                % \end{outerlist}

                % \href{http://www.phaseiv.com/}{\textbf{Phase IV Systems}} -- Huntsville, Alabama USA
                % \begin{outerlist}
                % \item[] \textit{Summer Hire}%
                %         \hfill \textbf{Summer 2005}
                % \begin{innerlist}
                % \item Supported Army Radar Operations Facility with hands-on operational testing of fielded
                %       Army radars (Sentinel Enhanced Target Range and Classification (ETRAC), Full Rate Production
                %       Option 5 (FRP5)) and HWIL simulation with injected threat profiles.

                % \item Collected Radar Cross Section (RCS) measurements of various threats for Stryker, HMMWV,
                %       and Helicopter mounted Active Protection Systems (APS).
                % \end{innerlist}

                % \item[] \textit{Co-op Student}%
                %         \hfill \textbf{Summer 2003; Spring, Fall 2004}
                % \begin{innerlist}
                % \item Supported field-testing and radar development for Stryker mounted APS.
                % \end{innerlist}
                % \end{outerlist}

                % \blankline
        \section{\textbf{Publications}}
                % \addbibresource{myreferences.bib}
                

                \nocite{*}
                \begin{innerlist}
                        \item \fullcite{10.1117/12.3043323}
                        \item \fullcite{10.1117/12.3041400}
                        \item \fullcite{Duckett2023}
                        \item \fullcite{10.1117/12.2649577}
                        \item \fullcite{9843596}
                \end{innerlist}
                
                

                % \printbibliography[title=\textbf{Publications}]
                
        \section{\textbf{Special Activities and Awards} }
        %
                \begin{outerlist}
                        \item[] \textit{NASA Awards}%
                        \begin{innerlist}
                                \item \textbf{2025 NASA Group Achievement Award} -- MGSS Collaborative Mission Control Evaluation and Testing Team
                                \item \textbf{2023 NASA Voyager Individual Achievement Award} -- Psyche DSOC Ground Data System Design, Development, Testing, and Deployment 
                                \item \textbf{2023 NASA Group Achievement Award} -- Exceptional System Engineering, Planning, Coordination, Software Development, and Testing of AMPCS Python 3 Migration
                                \item \textbf{2021 NASA Group Achievement Award} -- Outstanding Contributions to the Successful Development, Test, and Delivery of the Psyche Flight Software
                        \end{innerlist}
                        \item[] \textit{{Personal}}%
                        \begin{innerlist}
                                \item \href{http://www.auburn.edu/honors/college/}{Auburn University Honors College, 2006}
                                \item \href{http://www.eng.auburn.edu/organizations/HKN/}{Eta Kappa Nu -- Xi Chapter, 2005}
                                \item \href{http://www.eng.auburn.edu/organizations/SOA/}{Auburn University Solar Car Team, 2004}
                                \item \href{https://www.scouting.org/about/research/eagle-scouts/}{Boy Scouts of America Eagle Scout, 1999}\\\\
                        \end{innerlist}
                \end{outerlist}

                % \blankline
                % % \href{https://github.jpl.nasa.gov/stewartl}{github.jpl.nasa.gov / stewartl}
                % % \blankline
                % \href{https://github.com/stewalc}{github.com / stewalc}
                % \\(apologies, most of my efforts live in JPL enterprise github)


\end{document}

%%%%%%%%%%%%%%%%%%%%%%%%%% End CV Document %%%%%%%%%%%%%%%%%%%%%%%%%%%%%
